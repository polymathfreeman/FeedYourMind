\documentclass[12pt,4]{article}
\usepackage{fontspec}
\usepackage{xunicode}
\newfontfamily\cyrillicfont{Times New Roman}
\usepackage{polyglossia}
\usepackage{xltxtra}
\setmainlanguage{french}
\setotherlanguages{english}
\usepackage{fancyhdr}
\pagestyle{fancy}
\usepackage{graphics}
\usepackage{hyperref}
\hypersetup{
	colorlinks=true,
	citecolor=electricblue,
	filecolor=orange,
	linkcolor=red,
	urlcolor=blue
}
\author{Laurent Garnier}
\date{}
\title{Règle fondamentale de l'anglais n°1 avec Python 3 et \XeLaTeX}

\begin{document}

\begin{titlepage}

\begin{center}

{\Large {\sc Production d'un fichier TeX avec Python }}
\vspace{1cm}

\end{center}
\newpage

\vspace{2cm}

\begin{center}

BY

{\Large Laurent \textsc{Garnier} \par}

\vspace{2cm}

\end{center}
\begin{enumerate}
\end{enumerate}
\newpage

\begin{center}
\begin{enumerate}
\item {\sc Regardez ma playlist \href{https://youtube.com/playlist?list=PLO3S2CDkdJ9y45D8B87ROYq8FvGd_F_uQ}{Feed Your Mind Learn Python} : \url{https://youtube.com/playlist?list=PLO3S2CDkdJ9y45D8B87ROYq8FvGd_F_uQ}}\par
\end{enumerate}
\begin{enumerate}
\item {\sc Subscribe to my YouTube channel} \href{https://www.youtube.com/channel/UCIanf7BXjtmdfGKTvN5qtqA}{Polymath Freeman} : \url{https://www.youtube.com/channel/UCIanf7BXjtmdfGKTvN5qtqA}
\end{enumerate}

\end{center}

\end{titlepage}

\tableofcontents

\clearpage
\fancyhead[L]{\tt \url{https://lefturl.com}}
\fancyhead[R]{\tt \url{https://righturl.com}}
\renewcommand{\headrulewidth}{.2pt}
\renewcommand{\footrulewidth}{.4pt}
\section{The Rule}
\label{sec:org11a7b5a}
\textbf{La 3ème personne du singulier du présent} prend un \textbf{s}, que l'on  prononce toujours (seules exceptions : les auxiliaires de modalité)
L'oubli de ce s oralement comme à l'écrit est perçu par les anglophones comme une faute grossière.
\subsection{Examples}
\label{sec:org4c61b95}
\begin{description}
\item[{\href{https://fr.bab.la/conjugaison/anglais/play}{to play}}] he plays (prononcé comme un z)
\item[{\href{https://fr.bab.la/conjugaison/anglais/stop}{to stop}}] he stops (prononcé comme un s)
\end{description}
\section{Irrégularités}
\label{sec:orga23384d}
\subsection{Modification sonore}
\label{sec:org2823a0d}
\begin{description}
\item[{\href{https://fr.bab.la/conjugaison/anglais/have}{to have}}] he has (prononcé comme un z)
\item[{\href{https://fr.bab.la/conjugaison/anglais/be}{to be}}] he is (prononcé comme un z)
\item[{\href{https://fr.bab.la/conjugaison/anglais/do}{to do}}] he does (prononcé comme un z)
\item[{\href{https://fr.bab.la/conjugaison/anglais/go}{to go}}] he goes (prononcé comme un z)
\item[{\href{https://fr.bab.la/conjugaison/anglais/say}{to say}}] he says (prononcé comme un z)
\end{description}
\subsection{Ajout du es}
\label{sec:orgfd7621c}
\begin{description}
\item[{\href{https://fr.bab.la/conjugaison/anglais/cross}{to cross}}] he crosses (es prononcé iz)
\item[{\href{https://fr.bab.la/conjugaison/anglais/relax}{to relax}}] he relaxes (es prononcé iz)
\item[{\href{https://fr.bab.la/conjugaison/anglais/wash}{to wash}}] he washes (es prononcé iz)
\item[{\href{https://fr.bab.la/conjugaison/anglais/reach}{to reach}}] he reaches (es prononcé iz)
\end{description}
\subsection{Ajout du s avec le son iz}
\label{sec:orgee2fcc8}\begin{description}
\item[{\href{https://fr.bab.la/conjugaison/anglais/change}{to change}}] he changes
\item[{\href{https://fr.bab.la/conjugaison/anglais/judge}{to judge}}] he judges
\end{description}
\subsection{Les verbes terminés par y précédé d'une consonne}
\label{sec:org8062a1c}
\begin{description}
\item[{\href{https://fr.bab.la/conjugaison/anglais/carry}{to carry}}] he carries
\item[{\href{https://fr.bab.la/conjugaison/anglais/fly}{to fly}}] he flies
\end{description}
\subsection{Les verbes terminés par y précédé d'une voyelle}
\label{sec:org1606ffb}
\begin{description}
\item[{\href{https://fr.bab.la/conjugaison/anglais/buy}{to buy}}] he buys
\item[{\href{https://fr.bab.la/conjugaison/anglais/destroy}{to destroy}}] he destroys
\end{description}
\end{document}